

\documentclass[titlepage]{article}

\usepackage{tikz}
\usepackage{pgfplots}
\pgfplotsset{compat=1.18}
\pgfplotsset{hide xscale/.style={/pgfplots/xtick scale label code/.code={}}}
\pgfplotsset{hide yscale/.style={/pgfplots/ytick scale label code/.code={}}}
\usepackage{amsmath}
\usepackage{url}
\hyphenation{op-tical net-works semi-conduc-tor}	% Correct bad hyphenation
\usepackage{graphicx}  % For including figures and pictures
\usepackage{float}  % Used to fix location of images i.e.\begin{figure}[H]
\usepackage{fancyhdr}   % For headers and footers
\pagestyle{fancy}
\usepackage{titling}    % To reference title, author, and date
\usepackage{array}      % For fixed-width tables
\usepackage{appendix}   

% TITLE, AUTHOR, DATE, DOCUMENT NUMBER, STATUS,
\title{Redesign of the Visitor Center Solar Telescope}
\author{Remy Nguyen
    }% <-this % stops a space
\date{June 21, 2023}
\def\docnum{Assign Here}
\def\status{Draft}

% HEADERS AND FOOTERS
% If title is too long, write in text and use \\ to make line breaks
\fancyhead[L]{\textit{\thetitle} \\ \textit{NRAO Doc. \#: \docnum}}
\fancyhead[R]{\textit{\theauthor} \\ \thedate}

 % DOCUMENT STARTS HERE
\begin{document}

\begin{titlepage}
\begin{center}
     \vspace*{1cm}
     \includegraphics[width=5cm]{images/NRAO Logo Badge.png} \\
     \vspace*{0.5cm}
     \textbf{\Huge\thetitle} \\
     \vspace*{0.5cm}
     \large\docnum \\
     \huge Status: \status
     
\end{center}
\end{titlepage}

\tableofcontents
\thispagestyle{fancy}
\newpage
        
\section{Introduction}
Located behind the Visitor Center at the Very Large Array (VLA) is a telescope that converts RF solar energy into a DC voltage through amplification. It was originally constructed in 2013 and is no longer functioning as of June 2023. Weathered components and lack of sufficient documentation to repair the telescope to its original state has warranted a redesign of the entire system to be more appealing to visitors and resistant to wear.

\subsection{Purpose}


\subsection{Scope}
The heart of the telescope is an X-band feed horn antenna originally used to receive Voyager transmissions at 8.4 GHz; this is to be unchanged in the redesign. Additionally, DC power is supplied through a +28VDC power supply located inside the visitor center in order to eliminate radio frequency interference (RFI) caused by AC power and switching power supplies.

\section{Related Documents and Drawings}
\subsection{Applicable Documents}
The following documents may not be directly referenced herin, but may provide necessary context or supporting material.
\begin{center}
\begin{tabular}{|m{2cm}|m{7cm}|m{2.5cm}|} \hline
    Ref. No. & Document Title & Rev/Doc. No. \\ \hline
    AD01 & Title 1 & 0001 \\
    AD02 & Title 2 & 0002 \\
    \hline
\end{tabular}
\end{center}

\subsection{Reference Documents}
The following documents are referenced within this text:
\begin{center}
\begin{tabular}{|m{2cm}|m{7cm}|m{2.5cm}|} \hline
    Ref. No. & Document Title & Rev/Doc. No. \\ \hline
    AD01 & Title 1 & 0001 \\
    AD02 & Title 2 & 0002 \\
    \hline
\end{tabular}
\end{center}

\section{Heading 1}


\subsection{Heading 2}


\subsubsection{Heading 3}


\appendix
\section{Appendix Title}

\bibliographystyle{IEEEtran}
\bibliography{references}

\end{document}


