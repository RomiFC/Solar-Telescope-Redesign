\documentclass[titlepage]{article}

\usepackage{geometry}
\usepackage{tikz}
\usepackage{pgfplots}
\usepgfplotslibrary{fillbetween}
\pgfplotsset{compat=1.18}
\pgfplotsset{hide xscale/.style={/pgfplots/xtick scale label code/.code={}}}
\pgfplotsset{hide yscale/.style={/pgfplots/ytick scale label code/.code={}}}
\usepackage{amsmath}
\usepackage{url}
\hyphenation{}	        % Correct bad hyphenation
\usepackage{graphicx}   % For including figures and pictures
\usepackage{float}      % Used to fix location of images i.e.\begin{figure}[H]
\usepackage{fancyhdr}   % For headers and footers
\pagestyle{fancy}
\usepackage{lastpage}   % Allows referencing last page (for footer)
\usepackage{titling}    % To reference title, author, and date
\usepackage{array}      % For fixed-width tables
\usepackage[american]{circuitikz} % Used to draw circuit and block diagrams
\usepackage{appendix}   
\usepackage{colortbl}   % For coloring table cells
\usepackage{enumitem}   % For formatting enumerated lists
\usepackage[parfill]{parskip}               % Removes paragraph indentation, adds line break
\usepackage[hang, flushmargin]{footmisc}    % Removes footnote indentation
\setenumerate[1]{label={(\arabic*)}}
\usepackage[backend=biber,style=numeric,sorting=none]{biblatex}
\addbibresource{references.bib}
\usepackage{hyperref}   % Allows hyperlinks
\hypersetup{
  colorlinks   = true,  % Colors links instead of ugly boxes
  filecolor    = blue,  % Color for local hyperlinks
  linkcolor    = black, % Color for internal hyperlinks
  citecolor    = black  % Color for citations
}

% Global CircuiTikZ parameters
\ctikzset{sources/fill=red!20}
\ctikzset{chips/fill=cyan!20}
\ctikzset{electromechanicals/fill=blue!20}
\ctikzset{blocks/fill=green!20}
\ctikzset{amplifiers/fill=yellow!30}
\tikzset{amp/.append style={fill=yellow!30}}
\tikzset{twoport/.append style={fill=cyan!20}}

% Suppresses chktex warnings 
% chktex-file 1
% chktex-file 3
% chktex-file 8
% chktex-file 10
% chktex-file 17
% chktex-file 24
% chktex-file 36
% chktex-file 44
% chktex-file 46

% The following lines define a new command \nraocite that cites references in NRAO format RD0X by redefining \cite command to remove brackets. \nraoprecite includes a prenote.
\DeclareCiteCommand{\cite} [\mkbibemph{\emph{}}]
  {\usebibmacro{prenote}}
  {\usebibmacro{citeindex}%
   \printtext[bibhyperref]{\usebibmacro{cite}}}
  {\multicitedelim}
  {\usebibmacro{postnote}}
\DeclareCiteCommand*{\cite} [\mkbibemph\emph{}]
  {\usebibmacro{prenote}}
  {\usebibmacro{citeindex}
   \printtext[bibhyperref]{\usebibmacro{citeyear}}}
  {\multicitedelim}
  {\usebibmacro{postnote}}
\newcommand{\nraocite}[1]{[RD0\cite{#1}]}
\newcommand{\nraoprecite}[2][]{[RD0\cite{#2}{, #1}]}

% This command creates a link to the research directory in filehost
\newcommand{\filehost}{\href{//filehost/evla/techdocs/RFI/coopshare/remynguyen/SolarTelescope-ASWA/Solar-Telescope-Redesign/Research}{Filehost}}

% ASSIGN TITLE, AUTHOR, DATE, DOCUMENT NUMBER, STATUS, HERE
\title{Redesign of the Visitor Center Solar Telescope}
\author{Remy Nguyen
    }% <-this % stops a space
\date{June 21, 2023}
\def\docnum{[002.35.25.05.20-002]}
\def\status{\textcolor{red}{Draft}}

% HEADERS AND FOOTERS
\renewcommand{\headrulewidth}{0pt}
\fancyheadoffset[L,R]{2cm}
\fancyhead[L]{\vspace{-0.8cm}\includegraphics[width=2cm]{images/NRAO Logo Badge.png}}
\fancyhead[R]{
\renewcommand{\arraystretch}{1.4}
\renewcommand{\arrayrulewidth}{0.25pt}
\begin{tabular}{|m{6.7cm}|m{4.9cm}|m{3.4cm}|}
    \hline
        \textit{\textbf{Title:}} \thetitle &
        \textit{\textbf{Owner:}} \theauthor &
        \textit{\textbf{Date:}}  \thedate \\
    \hline
        \multicolumn{2}{|l|}{
            \textit{\textbf{NRAO Doc \#:} \docnum}  
        } &
        \textit{\textbf{Version:}} 1 \\
    \hline
\end{tabular}
}
\fancyfoot[C]{Page \textbf{\thepage} of \textbf{\pageref{LastPage}}}

 % DOCUMENT STARTS HERE
\begin{document}
\definecolor{nraoblue}{rgb}{0.776,0.851,0.945}  % Color of table headers
\setlength{\leftmargin}{1in}        % Sets margin
\setlength{\rightmargin}{1in}       % Sets margin
\setlength{\voffset}{-1.2in}        % Moves header up
\setlength{\headheight}{3.5cm}      % Defines header height
\setlength{\textheight}{591pt}      % Extends text/body size down
\setlength{\footskip}{60pt}         % Defines gap between body and footer

\thispagestyle{fancy}
\begin{center}
     \includegraphics[width=5cm]{images/NRAO Logo Badge.png} \\
     \vspace*{0.5cm}
     \textbf{\Huge\thetitle} \\
     \vspace*{0.5cm}
     \large\docnum \\
     \huge Status: \status \\
     \vspace*{1cm} \large
     % FIRST TABLE HERE
     \begin{tabular}{|m{6.93cm}|m{4.5cm}|m{2cm}|} \hline
        \rowcolor{nraoblue}
        \textbf{Prepared By} & \textbf{Organization} & \textbf{Date} \\ \hline
        R. Nguyen & NRAO Electronics Div. & 6/21/2023 \\ 
        \hline
    \end{tabular} \\
    \vspace*{1cm}
    \begin{tabular}{|m{3cm}|m{3.5cm}|m{6.93cm}|} \hline
        \rowcolor{nraoblue}
        \textbf{Approvals} & \textbf{Organization} & \textbf{Signatures} \\ \hline
    \end{tabular}
    \renewcommand{\arraystretch}{1.5}
    % CONTENT OF SECOND TABLE HERE
    \begin{tabular}{|m{3cm}|m{3.5cm}|m{6.93cm}|}
        T. Mobraten & NRAO Elec. Division &  \\ 
        \hline
        S. Wimbrow & NRAO Elec. Division &  \\ 
        \hline
        D. Schafer & NRAO Elec. Division &  \\ 
        \hline
    \end{tabular} \\
    \renewcommand{\arraystretch}{1}
    \vspace*{1cm}
    \begin{tabular}{|m{3cm}|m{3.5cm}|m{6.93cm}|} \hline
        \rowcolor{nraoblue}
        \textbf{Released By} & \textbf{Organization} & \textbf{Signature} \\ \hline
    \end{tabular}
    \renewcommand{\arraystretch}{1.5}
    % CONTENT OF THIRD TABLE HERE
    \begin{tabular}{|m{3cm}|m{3.5cm}|m{6.93cm}|}
        D. Schafer & NRAO Elec. Division &  \\ 
        \hline
    \end{tabular}
    \renewcommand{\arraystretch}{1}
\end{center}

% CHANGE RECORD
\newpage
\section*{Change Record}
\begin{center}
\renewcommand{\arraystretch}{1.2}
    \begin{tabular}{|m{1.5cm}|m{2.2cm}|m{2.5cm}|m{1.7cm}|m{5cm}|} \hline
        \rowcolor{nraoblue}
        Version & Date & Author & Affected\newline Section(s) & Reason\\ \hline \hline
    \end{tabular}
\renewcommand{\arraystretch}{1.6}
    \begin{tabular}{|m{1.5cm}|m{2.2cm}|m{2.5cm}|m{1.7cm}|m{5cm}|}
        1 & \thedate & R. Nguyen & All & Initial Draft \\ \hline
          &          &           &     &               \\ \hline
          &          &           &     &               \\ \hline
          &          &           &     &               \\ \hline
          &          &           &     &               \\ \hline
          &          &           &     &               \\ \hline
          &          &           &     &               \\ \hline
    \end{tabular}
\renewcommand{\arraystretch}{1}
\end{center}

\newpage
\tableofcontents
\listoffigures
\thispagestyle{fancy}
\newpage

\section{Introduction}

\subsection{Purpose}
Located behind the Visitor Center at the Very Large Array (VLA) is a radio telescope that converts RF solar energy into a DC voltage through amplification. It was originally constructed in 2013 and is no longer functioning as of June 2023. Weathered components and lack of sufficient documentation to repair the telescope to its original state has warranted a redesign of the entire system to be more appealing to visitors and resistant to wear.

\subsection{Scope}
This document will describe the design choices made in development of the new solar telescope. Calculations and experiments to support design decisions will be demonstrated along with constraints and requirements.


\section{Related Documents and Drawings}
\subsection{Applicable Documents}
The following documents may not be directly referenced herein, but may provide necessary context or supporting material.
\begin{center}
\begin{tabular}{|m{2cm}|m{7.5cm}|m{3.5cm}|} \hline
    \rowcolor{nraoblue}
    Ref. No. & Document Title & Rev/Doc. No.\\ \hline
    AD01 & Desiderata for Solar Observing with the EVLA & EVLAM\_70 \\ 
    \hline
    AD02 & EVLA Hardware Modifications in Support of Solar Observing & EVLAM\_72 \\
    \hline
\end{tabular}
\end{center}

\subsection{Reference Documents}
The following documents are referenced within this text:
\begin{center}
\renewcommand{\arraystretch}{1.2}
\begin{tabular}{|m{2cm}|m{7.5cm}|m{3.5cm}|} \hline
    \rowcolor{nraoblue}
    Ref. No. & Document Title & Rev/Doc. No.\\ \hline
    RD01 & \citefield{solartemp}{title} & \filehost \\ \hline
    RD02 & \citefield{aeh}{title} & DSOC Bookcase \\\hline
    RD03 & \citefield{tora}{title} & Privately Owned \\\hline
    RD04 & \citefield{xbandvla}{title} & \filehost \\\hline
    RD05 & \citefield{sfd1986}{title} & \filehost \\\hline
\end{tabular}
\renewcommand{\arraystretch}{1}
\end{center}

\section{System Overview}
The heart of the telescope is an conical corrugated X-band horn antenna originally used to receive Voyager transmissions at 8.4 GHz; this is to be unchanged in the redesign. Power is supplied through a 15 volt linear DC power supply located inside the visitor center in order to eliminate radio frequency interference (RFI) caused by AC power and switching power supplies.

As the telescope is pointed towards the various objects, RF energy is received by the antenna and amplified. A rectifying circuit converts the RF power into a DC voltage and further amplifies it before outputting to a voltmeter. The DC gain should be adjusted so that as the telescope is pointed directly at the sun, the voltmeter should raise noticeably from its value when pointed at a cold sky\footnote{Cold sky is defined when there are no notable RF emitting objects in the beamwidth. Experimentally, these readings seem to be equivalent to the receiver noise floor.}. When pointed at the ground or an object, the reading will increase to a maximum due reflections and filling of the antenna sidelobes. This is further explained in Section~\ref{sec:pwrplots}.

\subsection{Requirements}
Being a visitor center attraction, the device must be engaging and appealing to visitors. In order to achieve this, a transparent receiver to show electronics and a voltmeter that is driven by solar energy. The enclosure of the previous design was made of some sort of polycarbonate material which suffered from yellowing in the sun. Additionally, small computer fans with a mesh screen mounted directly to the enclosure were used to decrease operating temperature but allowed dust to accumulate over the years. In essence, the new design goals are as follows:
\begin{enumerate}
    \item Utilizes a transparent enclosure or panel to view electronics, ideally one which will not yellow or fade easily.
    \item Must be dustproof and resilient to accumulation of debris inside the enclosure while still venting heat from electronics.
    \item Electronics must be powered by a fixed DC linear power supply and easily serviceable.
    \item There must be a measurable voltage delta when pointing the telescope at various RF sources.
\end{enumerate}
The first two requirements were fulfilled by Tristen James in designing the enclosure. The following requirements will be discussed in this report.

\subsection{Architecture}
\begin{figure}[!ht]
    \begin{center}
        \begin{circuitikz}
            \draw(0, 0)
            node[
                rxantenna, 
                xscale=-1,
                label = {[xshift=-1.4cm, yshift=-0.5cm]X-band feed}
                ]{}
            to[amp = RF Gain Block, boxed] ++ (2, 0)
            to[bandpass = Filter] ++ (2,0)
            to[detector = Rectifier] ++ (2,0)
            to[amp = DC Gain Block, boxed] ++ (3,0)
            to[qvprobe = Voltmeter] ++ (2,0)
            -| ++ (0.5,-1) node[ground]{};
        \end{circuitikz}
    \caption{RF signal block diagram}\label{fig:rfblock}
    \end{center}
\end{figure}
Many of the RF components were either reused from the previous design or gathered at no cost. The RF gain block shown in Fig.~\ref{fig:rfblock} and~\ref{fig:rfgain} utilizes 3 low noise amplifiers (LNAs) and an attenuator. The first 2 LNAs are Mini-Circuits ZX60-06203ALN+ which provide a combined gain of 41 dB;\ they are followed by a ZX60-83LN-S+ which provides a mean gain of 15 dB in the filter bandwidth. These amplifiers are followed by an attenuator whose purpose will be described in Section~\ref{sec:tuning}. It is important to note that the receiver noise figure is about 2 dB -- a higher noise figure could result in an undetectable sun/sky delta.
\begin{figure}[!ht]
    \begin{center}
        \begin{circuitikz}
            \draw(0, 0)
            to[short, o-] ++ (0.5,0)
            to[amp, name=a1] ++ (2, 0)
            to[amp, name=a2] ++ (2, 0)
            to[amp, name=a3] ++ (2, 0)
            to[twoport, name=tp, fill=green!20] ++ (2, 0)
            to[short, -o] ++ (0.5,0);
            \draw (tp.center)
            node[resistorshape, anchor = center, rotate = 90, scale = 0.8]{};
            \draw (tp.center)
            ++ (0, 0.6)
            node[anchor = south]{-7 dB};
            \draw (a1.center) ++ (0, 0.6)
            node[anchor=south]{20.5 dB};
            \draw (a2.center) ++ (0, 0.6)
            node[anchor=south]{20.5 dB};
            \draw (a3.center) ++ (0, 0.6)
            node[anchor=south]{15 dB};
        \end{circuitikz}
    \caption{49 dB RF gain block}\label{fig:rfgain}
    \end{center}
\end{figure}

The bandpass filter does not have an attached datasheet but it resembles an equiripple filter and measured a 3 dB passband of about 7.8--8.9 GHz.

The rectifier and DC gain block are integrated into a PCB with SMA connectors. The former utilizes an Analog Devices ADL5902, providing a linear-in-decibel output that can be scaled by the choice of resistors on the board or the following gain element. The DC gain block utilizes an LM358 op amp with adjustable gain via potentiometer.

    

\subsection{Power Distribution}
\begin{figure}[!ht]
\begin{center}
\begin{circuitikz}
\ctikzloadstyle{legacy}
\ctikzset{bipoles/twoport/width=1}
\tikzset{twoport/.append style={fill=red!20}}
    \draw(0,0)[color=black]
    to[vsource, l_ = $V_{in}$, a^ = 15V] ++ (0,-3)
    node[tlground]{}
    (0, 0) -- ++ (1,0)
    % U1
    to[twoport, t=L7812, name=u1, l=U1] ++ (1.5,0)
    -- ++ (1,0) coordinate (12v)
    -- ++ (2,0)
    % U2
    to[twoport, t=L7805, name=u2, l=U2] ++ (1.5,0)
    -- ++ (1,0) coordinate (5v1)
    -- ++ (2,0) coordinate (5v2)
    -- ++ (2,0) coordinate (5v3)
    to[short, -o] ++ (1,0);
    % Amplifiers
    \draw(5v1.south)
    to[short, f = 150 mA] ++ (0,-2)
    node[plain mono amp, anchor=up, scale = 0.75, name = lna2]{LNA};
    \draw(5v2.south)
    to[short, f = 150 mA] ++ (0,-2)
    node[plain mono amp, anchor=up, scale = 0.75, name = lna3]{LNA};
    \draw(5v3.south)
    to[short, f = 94 mA] ++ (0,-2)
    node[plain mono amp, anchor=up, scale = 0.75, name = lna4]{LNA};
    \draw(lna2.in) to[open, o-o] (lna4.out);
    \draw (lna2.out) -- (lna3.in);
    \draw (lna3.out) -- (lna4.in);
    % ADL5902
    \draw(5v1.north)
    -- ++ (0, 0.2)
    to[short, f_ = 90 mA] ++ (0,1.75)
    node[qfpchip, anchor = pin 3, hide numbers, external pins width = 0, label = {[scale = 0.8]ADL5902}, name = U3]{}
    (U3.north) node[anchor = south]{U3};
    % Fan
    \draw(12v.south)
    to[short, f = 182 mA] ++ (0,-1.75)
    node[elmech, anchor = top, name = fan1]{M}
    (fan1.bottom) node[anchor=north]{Fan};
    \draw(12v.north)
    -- ++ (0, 0.3)
    to[short, f_ = 182 mA] ++ (0,1.75)
    node[elmech, anchor = bottom, name = fan2]{M}
    (fan2.top) node[anchor=south]{Fan};
    % Voltage Labels
    \draw (12v.north east)
    node[anchor = south west]{\color{red}12 V};
    \draw (5v2.north)
    node[anchor = south]{\color{red}5 V};
\end{circuitikz}
\caption{DC power distribution block diagram (Maximum ratings)}\label{fig:dcblock}
\end{center}
\end{figure}
As part of the effort to make the telescope more modern and visually appealing, the protoboard on the previous design was replaced with a dual layer PCB which contains the voltage regulators, the RF power detector IC, and dual noninverting op amps. The 15 volt supply is dropped to 12 volts and 5 volts via linear regulators, each of which powers different components. Figure~\ref{fig:dcblock} shows the maximum current draw of each component to ensure that the system does not exceed the 1.5 amp rating of the L78XX chips.


\section{Tuning System Gain}
\label{sec:tuning}
Figure~\ref{fig:vpcurve} shows the voltage-power curve of the ADL5902\footnote{Resistor values were changed to achieve 86.3 mV/dB slope: R6 = 1180$\Omega$ and R2 = 2000$\Omega$} at the telescope's operating frequency. Pointing the telescope at a high temperature object will produce an input power higher than pointing it at the cold sky; this dynamic range is represented by the dotted lines and filled section of the curve\footnote{The range has been enlarged for the purpose of demonstration. The true range is much smaller (Section~\ref{sec:pwrplots})}. Adding attenuation or RF gain will shift this area to the left or right respectively. Adding DC gain will increase the slope of this curve and vice versa.
\begin{figure}[!ht]
\begin{center}
    \begin{tikzpicture}
        \begin{axis}[
            title=$V_{out}$ vs. $P_{in}$ at 8.4 GHz,
            xlabel={$P_{in}$ (dBm)},
            ylabel={$V_{out}$ (V)},
            grid=major,
            xmin=-60, xmax=0,
            ymin=0, ymax=3,
            xtick={-60,-50,...,0},
            ytick={0,0.5,...,3},
            legend style = {
                align = left,
                cells = {anchor=east},
                legend pos=outer north east,
                }
                ]
                \path[name path=xaxis] (axis cs:-60,0) -- (axis cs:0,0);
                \addplot[smooth, blue, name path=f] table [x=Pin,y=Vout] {data/Vout vs Pin.txt};

                \addplot[opacity=0.2, green] fill between [of=f and xaxis, soft clip={domain=-47.5:-37.5}];
                \draw[dashed] (axis cs:-47.5,0) -- (-47.5, 1.25);
                \draw[dashed] (axis cs:-37.5,0) -- (-37.5, 1.25);
                \draw (axis cs:-42.5,1.25)
                node[anchor=south]{Ideal Range};

                \addplot[opacity=0.2, red] fill between [of=f and xaxis, soft clip={domain=-25:-15}];
                \draw[dashed] (axis cs:-25,0) -- (-25, 2.25);
                \draw[dashed] (axis cs:-15,0) -- (-15, 2.25);
                \draw (axis cs:-20,2.25)
                node[anchor=south]{Incorrect Range};
            \end{axis}
\end{tikzpicture}
\caption{Output voltage of the ADL5902-EVALZ as a function of input power.}
\label{fig:vpcurve}
\end{center}
\end{figure}
Ideally, the RF gain should be adjusted such that the low range is roughly 0 volts whereas the the high range is scaled to 3 volts by the DC gain block. In this case, the amplification from the DC gain block will have negligible effect on the low range. If, for some reason, granular attenuation is not possible, the DC gain will cause the nonzero low range to increase above 0 volts. A simple solution is presented in Fig.~\ref{fig:scalingcircuit} where the negative terminal of the voltmeter can be set to reference a nonzero voltage i.e\ the low range of the adjusted V-P curve.
\begin{figure}[!ht]
    \begin{center}
        \begin{circuitikz}
            \draw(0, 0)
            node[ampshape, anchor=east, boxed, fill=yellow!30](amp){}
            -- ++ (1, 0)
            to[qvprobe = Voltmeter] ++ (2.5,0)
            -- ++ (1,0) 
            coordinate (p1)
            to[R, l=R2] ++ (3,0)
            -| ++ (0, -0.5)
            node[cground]{};

            \draw(p1)
            to[R, l_=R1] ++ (0, 2)
            node[vcc]{15V};
            \draw[
                line width = 3pt,
                line cap = round,
                dash pattern = on 0pt off 3.5\pgflinewidth,
                ] (amp.west)
               ++ (-8pt, 0)
            -- ++ (-1, 0);
            \draw(amp.north)
            node[anchor=south]{DC Gain Block};
        \end{circuitikz}
    \caption{Optional output scaling circuit.}\label{fig:scalingcircuit}
    \end{center}
\end{figure}
In reality, the sensitivity of the receiver will prevent a perfect 0 to 3 volt scaling at all times. Thus, in order to ensure some amount of voltmeter deflection occurs despite environmental changes, a practical range 0.75--2.25 volts may be desired.


\section{Calculating Receiver Input Power}
The first step to determine RF gain will be to find the power delta as the antenna is pointed at the sun from a cold sky; note that this may vary daily and annually depending on solar cycle\nraoprecite[Fig.~2]{solartemp}. This calculation also assumes that solar flux density is constant across the entire frequency range of the receiver. This is a limitation of available information on solar flux density; in practice, it generally increases with frequency\nraocite{sfd1986}. Instantaneous channel power received by the antenna from a celestial source in watts, $P_R$, is defined by Eq.~\ref{eq:power}
\begin{equation}
    P_R = S A_e \Delta f
\label{eq:power}
\end{equation}
where $S$ is the source flux density in W m$^{-2}$ Hz$^{-1}$, $A_e$ is the effective aperture area in m$^2$, and $\Delta f$ is the receiver bandwidth in Hz\nraoprecite[eq.~(41-2)]{aeh}. To simplify calculations, we will focus on power spectral density (PSD) defined as $P(f)$ in dBm/Hz, ignoring the receiver bandwidth for now.
\begin{equation}
    P_{sun}(f) = S A_e
\label{eq:psd}
\end{equation}
Effective aperture can be calculated using measurements of the antenna as shown in Eq.~\ref{eq:ae}
\begin{equation} \label{eq:ae}
\begin{split}
    A_e &= \frac{\lambda^2}{4\pi}G \\
    &= \frac{\pi d^2}{4} \eta_a
\end{split}
\end{equation}
where $G$ is linear antenna gain, $d$ is the diameter of the circular horn at the aperture, and $\eta_a$ is the unitless aperture efficiency\nraoprecite[pg.~(15-27)]{aeh}\nraoprecite[eq.~(5.58)]{tora}. An ideal horn antenna has an $\eta_a$ of 0.522. The calculations for this design will use an approximate value of 0.5\footnote{The x-band horn measured an estimated aperture efficiency of 0.62$\pm$0.03 but only when mounted to the VLA dish\nraocite{xbandvla}. This design will assume the antenna to be a standard horn antenna.}. The antenna diameter at the opening measures 34 cm, resulting in an effective area of 0.0454 m$^2$ or -13.4 dB m$^2$.

Lastly, values for solar flux density $S$ must be found within the operating frequency. Ho \textit{et al.}\nraocite{solartemp} measured mean, minimum, and maximum values for solar flux density at 2800 MHz and 8800 MHz but only numbers for the former are shown in the text. However, they claim that solar flux density at 8800 MHz are higher by a factor of 2.17 on average, thus, the mean, minimum, and maximum values at 2800 MHz can be multiplied by 2.17 to achieve approximate values for $S$. Additionally, solar brightness temperature shown in~\nraoprecite[Fig.~5]{solartemp} can be used to calculate solar flux density via~\nraoprecite[eq.~(2)]{solartemp}. The former method shows higher dynamic range and is shown in Table~\ref{tab:sfd} in solar flux units\footnote{One solar flux unit (SFU) is equal to $10^{-22}$ W m$^{-2}$ Hz$^{-1}$.}.
\begin{table}[!ht]
\centering
\begin{tabular}{c|c|c|c}
    & $S_{\min}$ & $S_{\mu}$ & $S_{\max}$ \\ \hline
    2.8 GHz & 70 SFU & 150 SFU & 280 SFU \\
    8.8 GHz & 152 SFU & 326 SFU & 608 SFU
\end{tabular}
\label{tab:sfd}
\caption{Solar Flux Density $S$ at 8.8 GHz extrapolated by multiplying $S$ in~\nraocite{solartemp} at 2.8 GHz by a factor of 2.17.}
\end{table}

With values for $S$ and $A_e$ found, we can now calculate $P_{sun}(f)$ with equation~\ref{eq:psd} to be -181.6 dBm/Hz at $S_{\min}$. A problem arises when comparing this to the thermal noise PSD of -174 dBm/Hz at 300 K;\ because the power received by the sun is below the noise floor, it is insufficient to assume $P_R$ is independent of noise. The 7.6 dB delta between the noise floor and $P_{sun}(f)$ at $S_{\min}$ means that pointing the antenna at the sun will raise the antenna output power by a mere 0.7 dB at minimum. If a receiver noise figure of 2 dB is to be assumed, this output power delta decreases to 0.45 dB.
\begin{align*}
    \Delta P &= [P_{sun}(f)+P_{noise}(f)] - P_{noise}(f)\\
             &= 10 \log_{10} (10^{-17.4+0.2} + 10^{-18.16}) - (-174+2)\\
             &= 0.452 \text{ dB}
\end{align*}


\subsection{Antenna Power Measurements}
\label{sec:pwrplots}
In order to confirm these calculations, the antenna was connected to an Anritsu MS2720T/720 spectrum analyzer and pointed at various targets.
\begin{figure}[H]
\begin {center}
\begin{tikzpicture}
    \begin{axis}[
        title={Spectrum Analyzer Plot},
        xlabel=Frequency (MHz),
        ylabel={Power (dBm/300kHz)},
        grid=major,
        xmin=7400, xmax=9400,
        ymin=-87, ymax=-78,
        xtick={7400, 7800, ..., 9400},
        ytick={-86, -84, ..., -78},
        legend style = {
            align = left,
            cells = {anchor=west},
            legend pos=outer north east,
            }
            ]
        \addplot[orange!70] table [x index = 5, y index = 4] {data/FileName_1.txt};
        \addplot[blue!70] table [x index = 1, y index = 0] {data/FileName_1.txt};
        \addplot[red!80] table [x index = 3, y index = 2] {data/FileName_1.txt};
        \legend{Ground\\(-134.2 dBm/Hz), Sun\\(-136.3 dBm/Hz), Cold Sky\\(-136.9 dBm/Hz)}
    \end{axis}
\end{tikzpicture}
\caption{Signal trace with 40 dB of gain from two ZX60-06203ALN+ LNAs.}\label{fig:specan1}
\end {center}
\end{figure}
Mean channel power measurements on the Anritsu from 7.95 to 8.85 GHz show a 2.7 dB ground/sky delta and a 0.6 dB sun/sky delta.

\appendix

\end{document}


